% Metódy inžinierskej práce

\documentclass[twoside,slovak,a4paper]{coursepaper}
\usepackage[14pt]{extsizes}
\usepackage[slovak]{babel}
\usepackage[IL2]{fontenc}
\usepackage[utf8]{inputenc}
\usepackage{graphicx}
\usepackage{url}
\usepackage{hyperref}
\usepackage[utf8]{inputenc}
% \usepackage{times}

\pagestyle{headings} 

\title{Algoritmy na odporúčanie nových kontaktov a skupín v aplikácii LinkedIn\centering}

\author{Maksym Boiko\\[2pt]
	{ Slovenská technická univerzita v Bratislave}\\
	{ Fakulta informatiky a informačných technológií}\\
	{ \texttt{xboikom1@stuba.sk}}
	}

\date{\small 27. september 2024}


\begin{document}
% \begin{flushleft}
\maketitle

\section{Úvod}

Motivujte čitateľa a vysvetlite, o čom píšete. Úvod sa väčšinou nedelí na časti.~\cite{Oladipo:article}

Uveďte explicitne štruktúru článku. Tu je nejaký príklad.
Základný problém, ktorý bol naznačený v úvode, je podrobnejšie vysvetlený v časti~.
Dôležité súvislosti sú uvedené v častiach~ a~.
Záverečné poznámky prináša časť~.



% \section{Nejaká časť} \label{nejaka}



% \section{Iná časť} \label{ina}


% Môže sa zdať, že problém vlastne nejestvuje\cite{Coplien:MPD}, ale bolo dokázané, že to tak nie je~\cite{Oladipo:article}. Napriek tomu, aj dnes na webe narazíme na všelijaké pochybné názory\cite{PLP-Framework}. Dôležité veci možno \emph{zdôrazniť kurzívou}.



% \section{Svoja časť}

% \paragraph{Veľmi dôležitá poznámka.}


% \section{Dôležitá časť} \label{dolezita}




% \section{Ešte dôležitejšia časť} \label{dolezitejsia}



% \section{Záver} \label{zaver}

\thanks{Semestrálny projekt v predmete Metódy inžinierskej práce, ak. rok 2024/2025, vedenie: Richard Marko}

\bibliography{literatura}
\bibliographystyle{alpha} 
% \end{flushleft}
\end{document}
