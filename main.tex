\documentclass[twoside,slovak,a4paper]{coursepaper}
\usepackage[14pt]{extsizes}
\usepackage[slovak]{babel}
\usepackage[IL2]{fontenc}
\usepackage[utf8]{inputenc}
\usepackage{graphicx}
\usepackage{url}
\usepackage{hyperref}
\usepackage[utf8]{inputenc}
\usepackage{float}
\usepackage{wrapfig}
\usepackage{pdfpages}
% \usepackage{times}

\pagestyle{headings} 
\title{Odporúčacie algoritmy LinkedIn\centering}


\author{Maksym Boiko\\[2pt]
{ Slovenská technická univerzita v Bratislave}\\
{ Fakulta informatiky a informačných technológií}\\
{ \texttt{xboikom1@stuba.sk}}
}

\date{\small 27. september 2024}

\begin{document}


\maketitle

\section{Na čo sú určené?}
Odporúčacie algoritmy LinkedIn sú veľmi dôležité pri určovaní informácií, ktorý používatelia vidia vo svojich kanáloch.V tomto článku budete môcť pochopiť, ako fungujú odporúčacie algoritmy na platforme LinkedIn, a ako tieto znalosti efektívne využiť na propagáciu svojho profilu, spoločnosti alebo rychlo si najst povolanie.

\section{Actívny rozvoj spoločnosti} \label{rozvoj spoločnosti}

\begin{wrapfigure}{l}{0.5\textwidth}
	\includegraphics{STU-FIIT-zfv.png}
\end{wrapfigure}

Pre mnohých ľudí LinkedIn je zaroveň aj príležitosťou na profesionálny rozvoj, hľadanie práce a prácovné vzťahy.
Spoločnosť aktívne rozvíja svoje technológie s cieľom zlepšiť používateľskú skúsenosť a poskytovať relevantnejší obsah. V roku 2024 algoritmy LinkedIn sa výrazne zmenili a zameriavajú sa skôr na presnosť obsahu ako na viralitu. To znamená, že používatelia teraz môžu viac zamerať na zdieľanie odborných vedomostí a skúseností, namiesto toho, aby sa snažili o masový dosah.

~\cite{Oladipo:article}

\section{Užitočnosť odporúčacích algoritmov LinkedIn} \label{ina}
Odporúčacie algoritmy regulujú nielen príspevky, ktoré používateľ vidí vo svojom kanáli, ale aj ľudí, ktorých by on mohol poznať, ktorí majú podobné záujmy a povolania. Regulujú aj ponuky profesií, ktoré môžu byť pre človeka zaujímavé.

Pochopenie fungovania algoritmov LinkedIn vám umožní efektívnejšie propagovať vašu značku alebo spoločnosť sústredením sa na vytváranie hodnotného obsahu pre správne publikum.


% \section{Svoja časť}

% \paragraph{Veľmi dôležitá poznámka.}


% \section{Dôležitá časť} \label{dolezita}


% \section{Ešte dôležitejšia časť} \label{dolezitejsia}
 

\section{Záver} \label{zaver}
Odporúčacie algoritmy spoločnosti LinkedIn sa neustále zlepšujú. Platforma počíta s vašimi záujmami, interakciou s kolegami a aktuálnymi oblasťami vašej profesie. Na rozdiel od iných sociálnych sietí LinkedIn sa zameriava na zdieľanie vedomostí a odborných znalostí, a ne iba na komunikáciu medzi používateľmi.

\thanks{Semestrálny projekt v predmete Metódy inžinierskej práce, ak. rok 2024/2025, vedenie: Richard Marko}

\bibliography{literatura}
\bibliographystyle{alpha} 
% \end{flushleft}
\end{document}
