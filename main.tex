\documentclass[twoside,slovak,a4paper]{coursepaper}
\usepackage[14pt]{extsizes}
\usepackage[slovak]{babel}
\usepackage[IL2]{fontenc}
\usepackage[utf8]{inputenc}
\usepackage{graphicx}
\usepackage{url}
\usepackage{hyperref}
\usepackage[utf8]{inputenc}
\usepackage{float}
\usepackage{wrapfig}
\usepackage{pdfpages}
% \usepackage{times}

\pagestyle{headings} 
\title{Odporúčacie algoritmy LinkedIn\centering}


\author{Maksym Boiko\\[2pt]
{ Slovenská technická univerzita v Bratislave}\\
{ Fakulta informatiky a informačných technológií}\\
{ \texttt{xboikom1@stuba.sk}}
}

\date{\small 27. september 2024}

\begin{document}


\maketitle

\section{Na čo sú určené?}
Odporúčacie algoritmy LinkedIn sú veľmi dôležité pri určovaní informácií, ktorý používatelia vidia vo svojich kanáloch.V tomto článku budete môcť pochopiť, ako fungujú odporúčacie algoritmy na platforme LinkedIn, a ako tieto znalosti efektívne využiť na propagáciu svojho profilu, spoločnosti alebo rychlo si najst povolanie.

\section{Actívny rozvoj spoločnosti} \label{rozvoj spoločnosti}

Pre mnohých ľudí LinkedIn je zaroveň aj príležitosťou na profesionálny rozvoj, hľadanie práce a prácovné vzťahy.
Spoločnosť aktívne rozvíja svoje technológie s cieľom zlepšiť používateľskú skúsenosť a poskytovať relevantnejší obsah. V roku 2024 algoritmy LinkedIn sa výrazne zmenili a zameriavajú sa skôr na presnosť obsahu ako na viralitu. To znamená, že používatelia teraz môžu viac zamerať na zdieľanie odborných vedomostí a skúseností, namiesto toho, aby sa snažili o masový dosah.~\cite{Oladipo:article}

\section{Užitočnosť} \label{Užitočnosť}
Algoritmy regulujú mnohé aspekty odporúčaní s cieľom zlepšiť používateľskú skúsenosť a produktivitu. Hlavné z nich sú:

\begin{itemize}
	\item Odporúčania nových kontaktov, ktoré používateľ môže poznať
	\item Pomoc pri hľadaní zamestnania
	\item Výber obsahu pre informačný kanál
	\item Výber obsahu pre informačný kanál
	\item Odporúčania týkajúce sa kurzov odbornej prípravy a zručností
	\item Analýza a prognóza kariérneho vývoja
  \end{itemize}

Okrem toho, pochopenie fungovania algoritmov LinkedIn vám umožní efektívnejšie propagovať vašu značku alebo spoločnosť sústredením sa na vytváranie hodnotného obsahu pre správne publikum.


\section{Typy odporúčacích algoritmov} \label{Typy}
Existujú tri hlavné typy odporúčacích algoritmov: kolaboratívne filtrovanie, filtrovanie na základe obsahu a hybridné filtrovanie. 

\section{Ako algoritmy fungujú} \label{fungovanie}
\paragraph{Filtrovanie obsahu:}
Pozrime sa na odporúčania na príklade publikačných materiálov. Algoritmus LinkedIn používa rôzne faktory na určenie relevantnosti vášho príspevku pre vaše cieľové publikum.

Najprv keď niečo zverejníte bot zaradí váš obsah do kategórie \textbf{spam}, \textbf{nízka kvalita} alebo \textbf{vysoká kvalita} podľa hodnoty. ~\cite{Terez:article}

\paragraph{Relevantný obsah:}Platforma sa snaží vyzdvihnúť viac vedomostí a skúseností, o ktoré sa odborníci delia. Pre používateľov algoritmus určuje, ktoré odborné znalosti sú podstatné, pomocou identifikacii záujmov používateľa na základe informácií o jeho profile a činnosti.

LinkedIn využíva analytiku a strojové učenie na predpovedanie kariéry používateľa a pomáha mu plánovať jeho profesionálny rozvoj.	

\section{Uplatňovanie odporúčaní na rôzne časti platformy} \label{Uplatňovanie}


\section{Zbieranie a analýza údajov na účely odporúčaní} \label{analýza údajov}
 

\section{Záver} \label{zaver}
Odporúčacie algoritmy spoločnosti LinkedIn sa neustále zlepšujú. Platforma počíta s vašimi záujmami, interakciou s kolegami a aktuálnymi oblasťami vašej profesie. Na rozdiel od iných sociálnych sietí LinkedIn sa zameriava na zdieľanie vedomostí a odborných znalostí, a ne iba na komunikáciu medzi používateľmi.

\thanks{Semestrálny projekt v predmete Metódy inžinierskej práce, ak. rok 2024/2025, vedenie: Richard Marko}

\bibliography{literatura}
\bibliographystyle{alpha} 
% \end{flushleft}
\end{document}
